\documentclass[a4paper,12pt]{article}
\usepackage[utf8]{inputenc}
\usepackage{graphicx}
\usepackage{amsmath}
\usepackage{cite}
\usepackage{hyperref}
\usepackage{float}

\title{\textbf{Análisis Experimental de la Ecuación de Enfriamiento de Newton}}
\author{Estefania Cabal Aguado, Jonatan Galan Hernandez, \\
Angel Gabriel Maldonado Salinas, Aaron David Olguin Pichardo}
\date{\today}

\begin{document}

\maketitle

\begin{figure}[H]
    \centering
    \includegraphics[width=0.8\textwidth]{imagenPractica.jpg}
    \caption{Imagen de la practica.}
\end{figure}

\newpage
\tableofcontents
\newpage

\section{Introducción}
En este reporte, se presenta el análisis del enfriamiento de un cuerpo en función del tiempo, siguiendo la ecuación de enfriamiento de Newton. Se describe el procedimiento experimental, los datos recolectados y el ajuste de curva realizado mediante programación en Python.

\section{Ecuación de Enfriamiento de Newton}
La ecuación de enfriamiento de Newton establece que la velocidad de cambio de temperatura de un cuerpo es proporcional a la diferencia de temperatura entre el cuerpo y el ambiente:
\begin{equation}
    \frac{dT}{dt} = -k (T - T_{amb})
\end{equation}
donde $T$ es la temperatura del cuerpo, $T_{amb}$ la temperatura ambiente y $k$ es la constante de enfriamiento.

\section{Principio de Funcionamiento del Termómetro de Alcohol}
El termómetro de alcohol mide la temperatura a partir de la dilatación térmica del líquido dentro de un tubo capilar. Su precisión y respuesta dependen del tipo de alcohol y del diámetro del tubo.

\section{Procedimiento Experimental}
Se utilizó un termómetro de alcohol para medir la temperatura de un líquido caliente en intervalos regulares hasta alcanzar la temperatura ambiente. Los pasos seguidos fueron:
\begin{itemize}
    \item Calentar el líquido hasta 60°C.
    \item Medir la temperatura cada 5 minutos.
    \item Registrar los datos en una tabla.
\end{itemize}

\begin{figure}[H]
    \centering
    \includegraphics[width=0.6\textwidth]{procedimiento.jpg} % Imagen ilustrativa
    \caption{Imagen representativa de la medicion.}
\end{figure}

\section{Resultados Experimentales}
\begin{table}[H]
    \centering
    \begin{tabular}{|c|c|}
        \hline
        Tiempo (min) & Temperatura (°C) \\
        \hline
        0  & 60 \\
        5  & 52 \\
        10 & 48 \\
        15 & 44 \\
        20 & 39 \\
        25 & 38 \\
        30 & 35 \\
        35 & 33 \\
        40 & 32 \\
        45 & 31 \\
        50 & 29 \\
        55 & 28 \\
        60 & 27 \\
        65 & 26 \\
        70 & 25 \\
        \hline
    \end{tabular}
    \caption{Datos experimentales obtenidos.}
\end{table}

\section{Código en Python}
\begin{verbatim}
import numpy as np
from scipy.optimize import curve_fit
import matplotlib.pyplot as plt

def modelo(t, A, B, C):
    return A * np.exp(-B * t) + C

t_data = np.array([0, 5, 10, 15, 20, 25, 30, 35, 40, \
45, 50, 55, 60, 65, 70])
T_data = np.array([60, 52, 48, 44, 39, 38, 35, 33, 32, \
31, 29, 28, 27, 26, 25])

parametros, _ = curve_fit(modelo, t_data, T_data, p0=[35, 0.5, 25])
A_opt, B_opt, C_opt = parametros

plt.scatter(t_data, T_data, label="Datos experimentales")
t_fit = np.linspace(0, 70, 100)
T_fit = modelo(t_fit, A_opt, B_opt, C_opt)
plt.plot(t_fit, T_fit, label=f"Curva ajustada A: {A_opt:.2f}, \
B: {B_opt:.2f}, C: {C_opt:.2f}", color='red')
plt.xlabel("Tiempo (t)")
plt.ylabel("Temperatura (T)")
plt.legend()
plt.show()
\end{verbatim}

\section{Explicación del Uso de \texttt{curve\_fit}}
La función \texttt{curve\_fit} de la librería \texttt{scipy.optimize} permite ajustar un conjunto de datos a una función no lineal mediante el método de mínimos cuadrados. Se usó para obtener los parámetros óptimos de la ecuación:
\begin{equation}
    T(t) = A e^{-Bt} + C
\end{equation}

\section{Gráfica de Datos Experimentales y Curva Ajustada}
\begin{figure}[H]
    \centering
    \includegraphics[width=0.8\textwidth]{grafica.png} 
    \caption{Gráfica de los datos experimentales y la curva ajustada.}
\end{figure}

\section{Conclusiones}
El experimento permitió validar la ecuación de enfriamiento de Newton. Se observó un ajuste adecuado con la curva exponencial obtenida mediante Python. Posibles mejoras incluyen aumentar la cantidad de datos y reducir los errores de medición.

\section{Referencias}
\begin{thebibliography}{9}

    \bibitem{newton} Incropera, F.P., DeWitt, D.P. (2011). \textit{Fundamentos de Transferencia de Calor}. Pearson.

    \bibitem{metrologia} Bevington, P.R. (2003). \textit{Data Reduction and Error Analysis for the Physical Sciences}. McGraw-Hill.

    \bibitem{scipy} Virtanen, P. et al. (2020). "SciPy 1.0: Fundamental Algorithms for Scientific Computing in Python". Nature Methods.

    \bibitem{curvefit} SciPy. (n.d.). \textit{curve\_fit — SciPy v1.15.2 Manual}. Recuperado de  
    \url{https://docs.scipy.org/doc/scipy/reference/generated/scipy.optimize.curve_fit.html}

    \bibitem{joshi} Joshi, S. (2023, enero 30). \textit{Scipy scipy.optimize.curve\_fit Método}. Delft Stack. Recuperado de  
    \url{https://www.delftstack.com/es/api/scipy/scipy-scipy.optimize.curve_fit-method/}

\end{thebibliography}


\end{document}
